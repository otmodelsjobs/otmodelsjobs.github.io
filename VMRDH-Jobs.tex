% Options for packages loaded elsewhere
\PassOptionsToPackage{unicode}{hyperref}
\PassOptionsToPackage{hyphens}{url}
\PassOptionsToPackage{dvipsnames,svgnames,x11names}{xcolor}
%
\documentclass[
  letterpaper,
  DIV=11,
  numbers=noendperiod]{scrreprt}

\usepackage{amsmath,amssymb}
\usepackage{iftex}
\ifPDFTeX
  \usepackage[T1]{fontenc}
  \usepackage[utf8]{inputenc}
  \usepackage{textcomp} % provide euro and other symbols
\else % if luatex or xetex
  \usepackage{unicode-math}
  \defaultfontfeatures{Scale=MatchLowercase}
  \defaultfontfeatures[\rmfamily]{Ligatures=TeX,Scale=1}
\fi
\usepackage{lmodern}
\ifPDFTeX\else  
    % xetex/luatex font selection
\fi
% Use upquote if available, for straight quotes in verbatim environments
\IfFileExists{upquote.sty}{\usepackage{upquote}}{}
\IfFileExists{microtype.sty}{% use microtype if available
  \usepackage[]{microtype}
  \UseMicrotypeSet[protrusion]{basicmath} % disable protrusion for tt fonts
}{}
\makeatletter
\@ifundefined{KOMAClassName}{% if non-KOMA class
  \IfFileExists{parskip.sty}{%
    \usepackage{parskip}
  }{% else
    \setlength{\parindent}{0pt}
    \setlength{\parskip}{6pt plus 2pt minus 1pt}}
}{% if KOMA class
  \KOMAoptions{parskip=half}}
\makeatother
\usepackage{xcolor}
\setlength{\emergencystretch}{3em} % prevent overfull lines
\setcounter{secnumdepth}{5}
% Make \paragraph and \subparagraph free-standing
\ifx\paragraph\undefined\else
  \let\oldparagraph\paragraph
  \renewcommand{\paragraph}[1]{\oldparagraph{#1}\mbox{}}
\fi
\ifx\subparagraph\undefined\else
  \let\oldsubparagraph\subparagraph
  \renewcommand{\subparagraph}[1]{\oldsubparagraph{#1}\mbox{}}
\fi

\usepackage{color}
\usepackage{fancyvrb}
\newcommand{\VerbBar}{|}
\newcommand{\VERB}{\Verb[commandchars=\\\{\}]}
\DefineVerbatimEnvironment{Highlighting}{Verbatim}{commandchars=\\\{\}}
% Add ',fontsize=\small' for more characters per line
\usepackage{framed}
\definecolor{shadecolor}{RGB}{241,243,245}
\newenvironment{Shaded}{\begin{snugshade}}{\end{snugshade}}
\newcommand{\AlertTok}[1]{\textcolor[rgb]{0.68,0.00,0.00}{#1}}
\newcommand{\AnnotationTok}[1]{\textcolor[rgb]{0.37,0.37,0.37}{#1}}
\newcommand{\AttributeTok}[1]{\textcolor[rgb]{0.40,0.45,0.13}{#1}}
\newcommand{\BaseNTok}[1]{\textcolor[rgb]{0.68,0.00,0.00}{#1}}
\newcommand{\BuiltInTok}[1]{\textcolor[rgb]{0.00,0.23,0.31}{#1}}
\newcommand{\CharTok}[1]{\textcolor[rgb]{0.13,0.47,0.30}{#1}}
\newcommand{\CommentTok}[1]{\textcolor[rgb]{0.37,0.37,0.37}{#1}}
\newcommand{\CommentVarTok}[1]{\textcolor[rgb]{0.37,0.37,0.37}{\textit{#1}}}
\newcommand{\ConstantTok}[1]{\textcolor[rgb]{0.56,0.35,0.01}{#1}}
\newcommand{\ControlFlowTok}[1]{\textcolor[rgb]{0.00,0.23,0.31}{#1}}
\newcommand{\DataTypeTok}[1]{\textcolor[rgb]{0.68,0.00,0.00}{#1}}
\newcommand{\DecValTok}[1]{\textcolor[rgb]{0.68,0.00,0.00}{#1}}
\newcommand{\DocumentationTok}[1]{\textcolor[rgb]{0.37,0.37,0.37}{\textit{#1}}}
\newcommand{\ErrorTok}[1]{\textcolor[rgb]{0.68,0.00,0.00}{#1}}
\newcommand{\ExtensionTok}[1]{\textcolor[rgb]{0.00,0.23,0.31}{#1}}
\newcommand{\FloatTok}[1]{\textcolor[rgb]{0.68,0.00,0.00}{#1}}
\newcommand{\FunctionTok}[1]{\textcolor[rgb]{0.28,0.35,0.67}{#1}}
\newcommand{\ImportTok}[1]{\textcolor[rgb]{0.00,0.46,0.62}{#1}}
\newcommand{\InformationTok}[1]{\textcolor[rgb]{0.37,0.37,0.37}{#1}}
\newcommand{\KeywordTok}[1]{\textcolor[rgb]{0.00,0.23,0.31}{#1}}
\newcommand{\NormalTok}[1]{\textcolor[rgb]{0.00,0.23,0.31}{#1}}
\newcommand{\OperatorTok}[1]{\textcolor[rgb]{0.37,0.37,0.37}{#1}}
\newcommand{\OtherTok}[1]{\textcolor[rgb]{0.00,0.23,0.31}{#1}}
\newcommand{\PreprocessorTok}[1]{\textcolor[rgb]{0.68,0.00,0.00}{#1}}
\newcommand{\RegionMarkerTok}[1]{\textcolor[rgb]{0.00,0.23,0.31}{#1}}
\newcommand{\SpecialCharTok}[1]{\textcolor[rgb]{0.37,0.37,0.37}{#1}}
\newcommand{\SpecialStringTok}[1]{\textcolor[rgb]{0.13,0.47,0.30}{#1}}
\newcommand{\StringTok}[1]{\textcolor[rgb]{0.13,0.47,0.30}{#1}}
\newcommand{\VariableTok}[1]{\textcolor[rgb]{0.07,0.07,0.07}{#1}}
\newcommand{\VerbatimStringTok}[1]{\textcolor[rgb]{0.13,0.47,0.30}{#1}}
\newcommand{\WarningTok}[1]{\textcolor[rgb]{0.37,0.37,0.37}{\textit{#1}}}

\providecommand{\tightlist}{%
  \setlength{\itemsep}{0pt}\setlength{\parskip}{0pt}}\usepackage{longtable,booktabs,array}
\usepackage{calc} % for calculating minipage widths
% Correct order of tables after \paragraph or \subparagraph
\usepackage{etoolbox}
\makeatletter
\patchcmd\longtable{\par}{\if@noskipsec\mbox{}\fi\par}{}{}
\makeatother
% Allow footnotes in longtable head/foot
\IfFileExists{footnotehyper.sty}{\usepackage{footnotehyper}}{\usepackage{footnote}}
\makesavenoteenv{longtable}
\usepackage{graphicx}
\makeatletter
\def\maxwidth{\ifdim\Gin@nat@width>\linewidth\linewidth\else\Gin@nat@width\fi}
\def\maxheight{\ifdim\Gin@nat@height>\textheight\textheight\else\Gin@nat@height\fi}
\makeatother
% Scale images if necessary, so that they will not overflow the page
% margins by default, and it is still possible to overwrite the defaults
% using explicit options in \includegraphics[width, height, ...]{}
\setkeys{Gin}{width=\maxwidth,height=\maxheight,keepaspectratio}
% Set default figure placement to htbp
\makeatletter
\def\fps@figure{htbp}
\makeatother

\KOMAoption{captions}{tableheading}
\makeatletter
\@ifpackageloaded{bookmark}{}{\usepackage{bookmark}}
\makeatother
\makeatletter
\@ifpackageloaded{caption}{}{\usepackage{caption}}
\AtBeginDocument{%
\ifdefined\contentsname
  \renewcommand*\contentsname{Table of contents}
\else
  \newcommand\contentsname{Table of contents}
\fi
\ifdefined\listfigurename
  \renewcommand*\listfigurename{List of Figures}
\else
  \newcommand\listfigurename{List of Figures}
\fi
\ifdefined\listtablename
  \renewcommand*\listtablename{List of Tables}
\else
  \newcommand\listtablename{List of Tables}
\fi
\ifdefined\figurename
  \renewcommand*\figurename{Figure}
\else
  \newcommand\figurename{Figure}
\fi
\ifdefined\tablename
  \renewcommand*\tablename{Table}
\else
  \newcommand\tablename{Table}
\fi
}
\@ifpackageloaded{float}{}{\usepackage{float}}
\floatstyle{ruled}
\@ifundefined{c@chapter}{\newfloat{codelisting}{h}{lop}}{\newfloat{codelisting}{h}{lop}[chapter]}
\floatname{codelisting}{Listing}
\newcommand*\listoflistings{\listof{codelisting}{List of Listings}}
\makeatother
\makeatletter
\makeatother
\makeatletter
\@ifpackageloaded{caption}{}{\usepackage{caption}}
\@ifpackageloaded{subcaption}{}{\usepackage{subcaption}}
\makeatother
\ifLuaTeX
  \usepackage{selnolig}  % disable illegal ligatures
\fi
\usepackage{bookmark}

\IfFileExists{xurl.sty}{\usepackage{xurl}}{} % add URL line breaks if available
\urlstyle{same} % disable monospaced font for URLs
\hypersetup{
  pdftitle={VMRDH-Jobs},
  pdfauthor={Srirama Bhamidipati},
  colorlinks=true,
  linkcolor={blue},
  filecolor={Maroon},
  citecolor={Blue},
  urlcolor={Blue},
  pdfcreator={LaTeX via pandoc}}

\title{VMRDH-Jobs}
\author{Srirama Bhamidipati}
\date{2024-07-02}

\begin{document}
\maketitle

\renewcommand*\contentsname{Job index}
{
\hypersetup{linkcolor=blue}
\setcounter{tocdepth}{2}
\tableofcontents
}
\bookmarksetup{startatroot}

\chapter*{Preface}\label{preface}
\addcontentsline{toc}{chapter}{Preface}

\markboth{Preface}{Preface}

This pdf acts as a manual to understand the OmniTrans jobs, their
purpose, inputs and outputs.

\part{Standard Uitvoer}

\chapter{Matrix Bewerkingen}\label{matrix-bewerkingen}

This group of jobs deal with various matrix handling techniques.

\section{Fratar Methode}\label{fratar-methode}

\subsection{Purpose}

Look inside each tab to understand what you will get from this job.

\subsection{Inputs}

Following are the inputs to this job.

\begin{Shaded}
\begin{Highlighting}[]
\NormalTok{fratarTest}\AttributeTok{.source\_cube} \KeywordTok{=} \VerbatimStringTok{\textquotesingle{}2020\_KAL\textquotesingle{}} \CommentTok{\# Geef MatrixCube op (hier: 2016\_SMC)        }
\NormalTok{fratarTest}\AttributeTok{.matrix} \KeywordTok{=} \KeywordTok{[}\DecValTok{1}\NormalTok{,}\DecValTok{2}\NormalTok{,}\DecValTok{1}\NormalTok{,}\DecValTok{103}\KeywordTok{]}     \CommentTok{\# Geef Matrix (1 PER AANROEP!) (Hier Auto OS)}
\end{Highlighting}
\end{Shaded}

\subsection{Outputs}

Following are the outputs to this job.

\begin{Shaded}
\begin{Highlighting}[]
\NormalTok{fratarTest}\AttributeTok{.destination\_cube} \KeywordTok{=} \VerbatimStringTok{\textquotesingle{}FratarDemo\textquotesingle{}} \CommentTok{\# Resultaatcube }
\end{Highlighting}
\end{Shaded}

\subsection{Code}

Download the code.\href{../../first.rb}{matrixcompress.rb}

\section{Matrixritten op basis van
voorbeeldzone}\label{matrixritten-op-basis-van-voorbeeldzone}

\subsection{Purpose}

Some text explaining what the code does.

\subsection{Inputs}

Following are the inputs to this job.

\subsection{Outputs}

Following are the outputs to this job.

\subsection{Code}

Download the code.\href{../first.rb}{matrixcompress.rb}

\section{Percentage groei ten opzichte van
voorbeeldzone}\label{percentage-groei-ten-opzichte-van-voorbeeldzone}

\subsection{Purpose}

Some text explaining what the code does.

\subsection{Inputs}

Following are the inputs to this job.

\subsection{Outputs}

Following are the outputs to this job.

\subsection{Code}

Download the code.\href{../first.rb}{matrixcompress.rb}

\chapter{Matrix Compressies}\label{matrix-compressies}

\section{Purpose}

Some text explaining what the code does.

\section{Inputs}

Following are the inputs to this job.

\section{Outputs}

Following are the outputs to this job.

\section{Code}

Download the code.\href{../first.rb}{matrixcompress.rb}

\chapter{Voertuigprestaties}\label{voertuigprestaties}

\section{Purpose}

Some text explaining what the code does.

\section{Inputs}

Following are the inputs to this job.

\section{Outputs}

Following are the outputs to this job.

\section{Code}

Download the code.\href{../first.rb}{matrixcompress.rb}

\chapter{Thermopunten}\label{thermopunten}

\section{Purpose}

Some text explaining what the code does.

\section{Inputs}

Following are the inputs to this job.

\section{Outputs}

Following are the outputs to this job.

\section{Code}

Download the code.\href{../first.rb}{matrixcompress.rb}

\chapter{Skim Matrix Exports}\label{skim-matrix-exports}

\section{Purpose}

Some text explaining what the code does.

\section{Inputs}

Following are the inputs to this job.

\section{Outputs}

Following are the outputs to this job.

\section{Code}

Download the code.\href{../first.rb}{matrixcompress.rb}

\part{Routines}

This folder contains the crypted files.

\chapter{Bereikbaarheid}\label{bereikbaarheid}

\section{Purpose}

Some text explaining what the code does.

\section{Inputs}

Following are the inputs to this job.

\section{Outputs}

Following are the outputs to this job.

\section{Code}

Download the code.\href{../first.rb}{matrixcompress.rb}

\chapter{Selected Link Compress}\label{selected-link-compress}

\section{Purpose}

Some text explaining what the code does.

\section{Inputs}

Following are the inputs to this job.

\section{Outputs}

Following are the outputs to this job.

\section{Code}

Download the code.\href{../first.rb}{matrixcompress.rb}

\chapter{INEXDO}\label{inexdo}

\section{Purpose}

Some text explaining what the code does.

\section{Inputs}

Following are the inputs to this job.

\section{Outputs}

Following are the outputs to this job.

\section{Code}

Download the code.\href{../first.rb}{matrixcompress.rb}

\chapter{Milieu}\label{milieu}

\section{Purpose}

Some text explaining what the code does.

\section{Inputs}

Following are the inputs to this job.

\section{Outputs}

Following are the outputs to this job.

\section{Code}

Download the code.\href{../first.rb}{matrixcompress.rb}



\end{document}
